%!TEX root = ../thesis.tex
%*******************************************************************************
%*********************************** Systematic Uncertainties *********
%*******************************************************************************


\chapter{Systematic uncertainties}

\ifpdf
    \graphicspath{{chapter-uncertainties/Figs/Raster/}{chapter-uncertainties/Figs/PDF/}{chapter-uncertainties/Figs/}}
\else
    \graphicspath{{chapter-uncertainties/Figs/Vector/}{chapter-uncertainties/Figs/}}
\fi

Several sources of systematic uncertainties need to be considered in the following. As laid out in \cref{ch:statistics}, they enter the likelihood as nuisance parameters and can be interpreted as a loss of information on the signal strength parameter. In the following, they are separated into experimental uncertainties arising from finite detector resolution and object reconstruction and theoretical uncertainties due to modelling of the physics processes during simulation. 

\section{Experimental uncertainties}

Experimental uncertainties arise from the experimental methods used to derive the signal and background rate estimations. They are evaluated using up and down variations provided either as variational weights in the case of efficiency uncertainties, or as additional variational samples derived by performing the object reconstruction with varied parameters.

\subsection{Pile-up reweighting and luminosity}

The \gls{mc} events were largely already generated before the full data was taken, and therefore before the full pileup profile in data was known. For this reason, the number of average interactions $\braket{\mu}$ per bunch crossing in \gls{mc} is in general not identical to that in data, necessitating a reweighting procedure in \gls{mc}. In order to account for differences in the measured inelastic \textit{pp} cross sections and the one obtained from \gls{mc} simulation, a scale factor of 1.03 is applied before the reweighting procedure. As \gls{mc} samples are generated with integer of $\braket{\mu}$ only, the scale factor is applied to data instead. The uncertainty on the pileup reweighting is evaluated by varying the data scale factor by $\pm 0.04$ and deriving variational pileup weights.

As detailed in \cref{sec:lumi_datataking}, the total integrated luminosity relies on the measurement of the bunch luminosity which in turn needs precise measurements of the visible inelastic cross section $\sigma_\mathrm{vis}$ as well as the visible pile-up parameter $\mu_\mathrm{vis}$. Uncertainties on the measurement of the total recorded cross section are dominated by the uncertainties on $\sigma_\mathrm{vis}$ that is measured during special \gls{vdm} scans. For the full Run~2 dataset, an overall luminosity uncertainty of $\pm 1.7\%$, is considered for all \gls{mc} processes not normalised to data using a \gls{cr}, derived using the methods described in \reference\cite{ATLAS-CONF-2019-021}.

\subsection{Triggers}

As all selections considered in the analysis apply a minimum $\etmiss$ requirement of $\SI{240}{\GeV}$ where the $\etmiss$ triggers are fully efficient (see~\cref{sec:trigger_strategy}), a 2\% normalisation uncertainty correlated over all bins is considered. 

\subsection{Leptons}

A large number of uncertainties on electrons arises from energy scale and resolution measurements~\cite{EGAM-2018-01,PERF-2017-01}. They are assumed to be fully correlated in $\eta$ and are summed in quadrature, resulting in one nuisance parameter for the energy scale and one for the resolution. Similarly, uncertainties on muons arise from calibrations of the muon momentum scale and resolution~\cite{PERF-2015-10}. They are evaluated using variations in the smearing of the \gls{id} and \gls{ms} tracks as well as the momentum scale, resulting in a total of five nuisance parameters entering the likelihood. Additional lepton uncertainties considered in the following originate from reconstruction, identification and isolation efficiencies, as well as track-to-vertex association and bad muon identification efficiencies. The latter two are only considered for muons.

\subsection{Jets}

The calibration of the jets to the absolute \gls{jes} is subject uncertainties arising \eg from the \textit{in situ} measurements, pile-up effects or flavour-dependence~\cite{Aad:2020flx}, encoded in a large set of 125 parameters. The full detail contained in the complete set of uncertainty components offers far greater statistical precision than needed for the following analysis. As the majority of the parameters (a total of 98) stems from \textit{in situ} measurements, an eigenvector decomposition is performed on the covariance matrix of these components~\cite{ATL-PHYS-PUB-2015-014}, allowing to determine the 15 principal orthogonal components (including a residual term adding the remaining terms in quadrature), with minimal loss in bin-by-bin correlation information. Five additional parameters evaluating uncertainties arising from \textit{in situ} $\eta$-intercalibrations of forward jets with respect to central jets are kept separate due to their two-dimensional dependence on $\pt$ and $\eta$~\cite{Aad:2020flx}. Effects from pile-up are described by four nuisance parameters. Uncertainties arising from differing detector responses to gluon- and quark-initiated jets as well as flavour-related differences are accounted for by two more nuisance parameters. Uncertainties from jets that are not contained in the calorimeters and \textit{punch-through} into the \gls{ms} are evaluated with an additional parameter. A last parameter encodes the uncertainty arising from the calibration of \gls{mc} samples reconstructed using \textsc{ATLFAST-II} instead of the full detector simulation.

Systematic uncertainties on the \gls{jer} arise from measured differences between data and \gls{mc} simulation, noise from pile-up, and \textit{in situ} measurements of the jet $\pt$ imbalance. A similar eigenvector decomposition as for part of the \gls{jes} uncertainties is used, reducing the set of nuisance parameters considered in the following to 13~\cite{Aad:2020flx}. Finally, uncertainties related to the efficiency of jet vertex tagging are evaluated using a weight systematic.

\subsection{Flavour tagging}
 
 Uncertainties on the flavour tagging efficiency originate \eg from modelling uncertainties as well as uncertainties on the reconstruction of physics objects. As for the \gls{jer} and \gls{jes} the full set of nuisance parameters that would in principle need to be included in order to consider the full bin-by-bin correlations and $\pt$ and $\eta$ dependence of the uncertainties, is reduced to a more manageable size using an eigenvector decomposition. This leads to a total of five nuisance parameters encoding uncertainties on the \textit{b}-tagging efficiency, c-jet and light-jet mistagging rate as well as the extrapolation to high-$\pt$ jets~\cite{FTAG-2018-01, PERF-2016-05}.  
 
\subsection{Missing transverse energy}

The uncertainties on $\etmiss$ are evaluated using the systematic variations of all calibrated objects as inputs to the $\etmiss$ calculation. Additional uncertainties arise from the calculation of the track soft term. In the following, uncertainties on the soft term scale and resolution are considered, resulting in one nuisance parameter for the soft term scale and two nuisance parameters (corresponding to the perpendicular and parallel components) for the soft term resolution uncertainty. All track soft term uncertainties are derived by comparing simulation to $Z\rightarrow\mu\mu$ events~\cite{PERF-2016-07}. Contributions to the \gls{jes} uncertainty due to pile-up is considered 	

\section{Theoretical uncertainties}

As discussed in~\cref{sec:mc_simulation}, due to finite order calculations, the different steps of the \gls{mc} simulation generally introduces a certain number of unphysical scales and parameters. In order to quantify the uncertainties arising from the ad-hoc values of these, the \gls{mc} simulation generally needs to be re-run with systematically varied parameter values. Since varied \gls{mc} simulation parameters affect the event kinematics even before reconstruction and calibration, it is computationally very expensive to produce a full set of variations for each \gls{mc} sample used in the nominal analysis.

In the following, multiple approaches are used to derive the theory uncertainties. For some of the variational samples, the full \gls{mc} simulation chain was run with reduced statistics or different samples produced with a different set of \gls{mc} generators and tunes were available. For others, variations were already processed during the initial \gls{mc} simulation of the nominal sample and subsequently stored as variational weights. Finally, some of the variational samples were simulated at truth-level (\ie without detector simulation). The latter approach was used especially in the case of \gls{susy} signal samples, where theory uncertainties are expected to only have a minor impact and a full simulation of all parameter variations would be computationally unfeasible. 

For background processes that are normalised to data in a dedicated \gls{cr}, the theory uncertainties are evaluated on the transfer factors. For a process $p$, a control region CR$_i$, and a destination region R$_j$ (either a \gls{sr} or a \gls{vr}), the transfer factor reads
\begin{equation}
	f_p(\mathrm{CR}_i\rightarrow \mathrm{R}_j) = \frac{N^\mathrm{MC}_p(\mathrm{R}_j)}{N^\mathrm{MC}_p(\mathrm{CR}_i)},
\end{equation}
where $N^\mathrm{MC}_p(\mathrm{R}_j)$ and $N^\mathrm{MC}_p(\mathrm{CR}_i)$ are the expected event rates for the process $p$ in CR$_i$ and R$_j$, respectively. The systematic uncertainty on the transfer factor is then given by
\begin{equation}
	\Delta f_p^\mathrm{syst} = \frac{f_p^\mathrm{variation}}{f_p^\mathrm{nominal}} - 1,
\end{equation}
with $f_p^\mathrm{variation}$ and $f_p^\mathrm{nominal}$ the transfer factors from the variational and nominal samples, respectively. If the \gls{mc} samples used for deriving the variational and nominal transfer factors are statistically independent, a statistical component of the uncertainty is derived using the individual statistical uncertainties on the background estimate,
\begin{equation}
	\Delta f_p^\mathrm{stat} = (\Delta f_p^\mathrm{syst} + 1 ) \sqrt{\sum_{n\in\boldsymbol{N}}(\frac{\sigma_n}{n})^2},
\end{equation}
 where $n$ runs over the set of expected event rates and $\sigma_n$ is the absolute \gls{mc} statistical uncertainty associated to each expected event rate $n$. In the following the control region used to evaluate the uncertainties on the transfer factors is taken to be the sum of all \glspl{cr} introduced in~\cref{sec:control_regions}. This not only significantly improves the statistics in the region used for normalisation, but also results in a consistent treatment across all theoretical uncertainties on all relevant processes.
 
 For other backgrounds directly estimated from \gls{mc} simulation, the systematic uncertainty on the expected event rate in each region R$_i$ is given by
 \begin{equation}
 	\Delta n_p^\mathrm{syst}(\mathrm{R}_i) = \frac{n_p^\mathrm{syst}(\mathrm{R}_i)n_p^\mathrm{nominal}(\mathrm{P})}{n_p^\mathrm{nominal}(\mathrm{R}_i)n_p^\mathrm{syst}(\mathrm{P})} - 1,
 \end{equation}
 where the region P is a so-called \textit{loose preselection} with minimal analysis selection criteria used for normalisation of the event rates to be compared. If not otherwise indicated the loose preselection used for normalisation requires exactly one isolated lepton, 2--3 jets of which at least one is \textit{b}-tagged, $\etmiss>\SI{220}{\GeV}$ and $\mt>\SI{50}{\GeV}$.
 
 Apart from the hard scattering and parton showering uncertainties on top processes, all other theoretical uncertainties enter the likelihood as asymmetric correlated shape uncertainties. The hard scattering and parton showering uncertainties on top processes described below are estimated using \gls{mc} generator comparisons and thus need to be symmetrised. The shape information is however still kept, \ie the uncertainties are not one-sided.
 
 \subsection{Background}
 
 \subsubsection{$\boldsymbol{\ttbar}$ and single top}
 
 Theory uncertainties on the estimate of $\ttbar$ and single top processes arise for example from the simulation of the hard scattering between the interacting partons. These are evaluated by comparing the nominal \gls{mc} samples generated using \textsc{Powheg} and \textsc{Pythia8} with alternative samples generated using \textsc{MadGraph\_aMC@NLO} and \textsc{Pythia8}. An uncertainty resulting from the hadronisation and fragmentation scheme chosen in \textsc{Pythia8} is estimated by comparison to a \gls{mc} sample generated using \textsc{Powheg} and \textsc{Herwig++}~\cite{Herwig:2015jjp}. Uncertainties arising from \gls{isr} are evaluated at full reconstruction level by varying up and down by a factor of two the unphysical renormalisation $\mu_\mathrm{R}$ and factorisation $\mu_\mathrm{F}$ scales as well as the parameters controlling the showering and \gls{me}+\gls{ps} matching~\cite{ATL-PHYS-PUB-2016-004}. Likewise, uncertainties arising from simulation of \gls{fsr} are estimated by varying the effective coupling $\alpha_s^{\mathrm{FSR}}$~\cite{ATL-PHYS-PUB-2016-004}. 
 
 Uncertainties also originate from the \gls{PDF} set used when simulating the nominal \gls{mc} sample. As detailed in~\cref{tab:mc_generators}, the \textsc{NNPDF3.0NLO} is used for the simulation of both $\ttbar$ and single top processes. An envelope around the variational expected event rates obtained from the \textsc{NNPDF3.0NLO} uncertainties are used to compute an uncertainty on the transfer factor.
 
 Beyond \gls{lo} single top production diagrams, interference appears between $Wt$ and $\ttbar$ production. Two approaches are commonly used to try and isolate the $Wt$ channel: \gls{dr} and \gls{ds}~\cite{Frixione:2008yi}. While the former removes all diagrams in the \gls{nlo} $Wt$ amplitude that are doubly resonant (meaning that they involve an intermediate $t$ quark that can be on-shell), the latter introduces subtraction terms in the \gls{nlo} $Wt$ cross section cancelling the $\ttbar$ contribution~\cite{Frixione:2008yi}. As the \gls{dr} scheme is used for estimating the event rate of the $Wt$ channel in the analysis, a comparison with an estimation using the \gls{ds} scheme allows to derive an uncertainty associated to the interference.
  
 \subsubsection{$\boldsymbol{W/Z+\mathrm{jets}}$}
 
 For $W/Z+\mathrm{jets}$ processes, simulated using \textsc{Sherpa 2.2.1}, four different unphysical scales can be varied in order to investigate uncertainties on the modelling. The renormalisation $\mu_\mathrm{R}$ and factorisation $\mu_\mathrm{F}$ scales are both independently and together varied up and down by a factor of two, resulting in a total of seven combined variations. Three envelopes are determined from varying only $\mu_\mathrm{R}$, only $\mu_\mathrm{F}$ or $\mu_\mathrm{R}$ and $\mu_\mathrm{F}$ together, allowing to determine three separate uncertainties. The CKKW \gls{me}+\gls{ps} matching scheme also uses an unphysical scale for determining the overlap between jets from the \gls{me} and the \gls{ps}. The nominal value of $\SI{20}{\GeV}$ for the merging scale is varied to $\SI{30}{\GeV}$ and $\SI{15}{\GeV}$ for the up and down systematic variations, respectively. Finally, the scale used for resummation of soft gluon emission, $\mu_\mathrm{QSF}$ is varied up and down by a factor of two.
 
 An additional uncertainty arises from the choice of \gls{PDF} set used for simulating $W/Z+\mathrm{jets}$. It is evaluated by propagating the \gls{PDF} error set (containing slightly different parameterisations of the \gls{PDF}) to the analysis observables. Uncertainties due to the choice of $\alpha_s = 0.118$ for fitting the \glspl{PDF} are estimated by comparing with variations using $\alpha_s(m_Z) = 0.119$ and $\alpha_s(m_Z) = 0.117$, and added in quadrature to the previous \gls{PDF} uncertainty.
 
 As $Z+\mathrm{jets}$ is not normalised to data in a dedicated \gls{cr} but to its nominal \gls{sm} cross section, an additional normalisation uncertainty corresponding to the uncertainty on the cross section is thus considered.
  
 \subsubsection{Other backgrounds}
 
 For diboson, multiboson and $\ttbar+V$ processes, uncertainties arising from the unphysical scales $\mu_\mathrm{F}$, $\mu_\mathrm{R}$ as well as $\mu_\mathrm{QSF}$ and the CKKW \gls{me}+\gls{ps} matching scale are considered using the same prescription described above for $W/Z+\mathrm{jets}$. For these three processes as well as for the other minor backgrounds $V+h$ and $\ttbar+h$, an additional uncertainty on the \gls{sm} cross section used for normalisation is considered.
  
 \subsection{Signal}\label{sec:signal_theory_uncertainties}
 
 Theoretical uncertainties on the \gls{susy} signal processes arise from the unphysical factorisation, renormalisation and CKKW-L \gls{me}+\gls{ps} merging scales. These are evaluated using a similar procedure as for background processes, varying the different scales up and down by a factor of two and comparing the expected signal rates. An additional uncertainty on \gls{ps} originating from the chosen \textsc{Pythia8} tune is estimated by varying up and down the chosen value for $\alpha_s^\mathrm{ISR}$.    
 
 As detailed in~\cref{sec:signal_samples}, the cross section of electroweakino pair production is calculated using \textsc{Resummino}. A theoretical uncertainty on the cross section is considered in the following, but does not enter the statistical fit procedure as nuisance parameter. Instead, in addition to the set of observed CL$_s$ values using the nominal cross section, two additional \textit{variational} sets are derived using signal cross sections fixed at their $\pm 1\sigma$ variations. This allows to draw a cross section uncertainty band on the observed exclusion contour. 

Due to the large number of \gls{mc} samples, all theory uncertainties on \gls{susy} signal processes are evaluated at truth-level only.   As the \glspl{vr} typically have relatively low signal contamination and thus low signal \gls{mc} statistics available for evaluating theory uncertainties, requirements on observables with negligible impact on the shapes of the theoretical uncertainties are loosened. In the on-peak \glspl{vr}, the requirements loosened are $\mt>\SI{60}{\GeV}$ and $\etmiss\SI{140}{\GeV}$. The same loosened selection is applied in \glspl{sr} in cases where \gls{mc} statistical uncertainty is too high for a reliable estimation of the theoretical uncertainties. In the off-peak \glspl{vr}, the requirements loosened are $\mt>\SI{60}{\GeV}$ and $\etmiss>\SI{60}{\GeV}$ and $\mct>\SI{60}{\GeV}$. Overall, the theoretical uncertainties on the expected signal rate range from about 10\% in phase space regions with large mass splitting between $\charg/\neutr$ and $\lsp$ to about 25\% in regions with small mass splittings.\improvement{Plots from HF numbers?}
 
 
\section{Impact on signal regions}

\Cref{tab:systematics_summary} shows a breakdown of the dominant systematic uncertainties on the background prediction in the \glspl{sr}, obtained after a background-only fit in the \glspl{cr} with subsequent extrapolation to the \glspl{sr}. The total uncertainty in the \glspl{sr} amounts to 15\% in SRLM and increases to 25\% in SRMM and 34\% in SRHM. Theoretical uncertainties have the largest contribution to the total uncertainty. For SRLM, the largest uncertainty originates from the $\ttbar$ parton shower uncertainty (10\%), while for SRMM (SRHM) the single top generator uncertainties are the largest ones with 10\% (21\%). Theoretical uncertainties on $\wjets$ and other minor backgrounds have only small to negligible effects. The experimental uncertainties in general have less impact on the total uncertainty than the theoretical ones, with the largest experimental uncertainties contributing only 5--10\% depending on the \gls{sr}. The dominant experimental uncertainties arise from the \gls{jes} and \gls{jer} as well as $\etmiss$ modelling and pile-up. The \gls{ms} statistical uncertainties contribute 5--18\% depending on the \gls{sr}.  

\begin{table}
\begin{center}
\setlength{\tabcolsep}{0.0pc}
\resizebox{\textwidth}{!}{
\begin{tabular*}{\textwidth}{lccc}
\toprule
\textbf{Signal Region}                                    & SRLM            & SRMM            & SRHM            \\
\midrule
Total background expectation             &  $27$        &  $8.6$        &  $8.1$       \\
\midrule
Total uncertainty               & $\pm 4\ [15\%]~$        & $\pm 2.2\ [25\%] $        & $~\pm 2.7\ [34\%] $             \\
\midrule
\multicolumn{4}{c}{Theoretical systematic uncertainties}\\
\midrule
$\ttbar$          & $~~\pm 2.6\ [10\%] $          & $\pm 0.6\ [7\%]~ $          & $~\pm 0.33\ [4\%] $       \\
Single top          & $~~~\pm 0.8\ [2.7\%] $          & $~\pm 1.1\ [12\%] $          & $~\pm 1.9\ [23\%] $       \\
$W$+jets         & $~~~~~\pm 0.23\ [0.9\%] $          & $~~~~\pm 0.07\ [0.8\%] $          & $~~~~\pm 0.19\ [2.3\%] $       \\
Other backgrounds      & $~~~~~\pm 0.13\ [0.5\%] $          & $~~~~\pm 0.15\ [1.7\%] $          & $~~~~\pm 0.08\ [1.0\%] $       \\
\midrule
\multicolumn{4}{c}{MC statistical uncertainties}\\
\midrule
MC statistics         & $\pm 1.7\ [6\%] $          & $~\pm 1.1\ [13\%] $          & $~\pm 1.2\ [14\%] $       \\
\midrule
\multicolumn{4}{c}{Uncertainties in the background normalisation}\\
\midrule
Normalisation of dominant backgrounds         & $\pm 1.3\ [5\%] $          & $~\pm 1.6\ [18\%] $          & $~\pm 1.3\ [16\%] $       \\
\midrule
\multicolumn{4}{c}{Experimental systematic uncertainties}\\
\midrule
$\met$/JVT/pile-up/trigger         & $\pm 1.8\ [7\%] $          & $\pm 0.4\ [4\%]~ $          & $\pm 0.4\ [5\%]~ $       \\
Jet energy resolution         & $\pm 1.6\ [6\%] $          & $\pm 0.5\ [6\%]~ $          & $\pm 0.4\ [5\%]~ $       \\
$b$-tagging         & $\pm 1.1\ [4\%] $          & $~~~~\pm 0.29\ [3.4\%] $          & $~~~~\pm 0.13\ [1.5\%] $       \\
Jet energy scale         & $~~~\pm 0.9\ [3.2\%] $          & $~\pm 0.9\ [10\%] $          & $~\pm 0.29\ [4\%] $       \\
Lepton uncertainties         & $~~~~~\pm 0.32\ [1.2\%] $          & $~~~~\pm 0.09\ [1.0\%] $          & $~~~~\pm 0.19\ [2.3\%] $   \\
\bottomrule
\end{tabular*}}
\end{center}
\caption[Breakdown of uncertainty in background estimates]{
Breakdown of the dominant systematic uncertainties in background estimates in the various exclusion signal regions ($\mct$ bins summed up).
As the individual uncertainties can be correlated, they do not necessarily add up in quadrature to 
the total background uncertainty. The percentages show the size of the uncertainty relative to the total expected background. Table adapted from \reference\cite{SUSY-2019-08}.}
\label{tab:systematics_summary}
\end{table}




