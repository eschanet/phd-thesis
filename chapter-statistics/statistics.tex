%!TEX root = ../thesis.tex
%*******************************************************************************
%*********************************** Statistics *********
%*******************************************************************************


\chapter{Statistical data analysis}\label{ch:statistics}

\ifpdf
    \graphicspath{{chapter-statistics/Figs/Raster/}{chapter-statistics/Figs/PDF/}{chapter-statistics/Figs/}}
\else
    \graphicspath{{chapter-statistics/Figs/Vector/}{chapter-statistics/Figs/}}
\fi

\glsreset{pdf}

Statistical models are used in order to order to quantify the correspondence between theoretical predictions and the experimental observations in searches for \gls{bsm} physics. This chapter introduces the statistical concepts, methods and formulae used in this work for statistical inference. A frequentist approach to statistics is employed, interpreting probabilities as the frequencies of the outcomes of repeatable experiments that may either be real, based on computer simulations, or mathematical abstraction~\cite{pdg2020,Cranmer:2015nia}. The ensuing description largely follows~\cite{Cranmer:2015nia, Cowan:2010js}

\section{The likelihood function}
 
In measurements in high energy physics, a \textit{statistical model} $f(\boldsymbol{x}\vert\boldsymbol{\phi})$ is a parametric family of \glspl{pdf} describing the probability of observing data $\boldsymbol{x}$ given a set of model parameters $\phi$ that typically describe parameters of the physical theory or unknown detector effects. The \textit{likelihood function} $L(\boldsymbol{\phi})$ is then numerically equivalent to $f(\boldsymbol{x}\vert\boldsymbol{\phi})$ with $\boldsymbol{x}$ fixed. As opposed to the \gls{pdf} $f(\boldsymbol{x})$ which describes the value of $f$ as a function of $\boldsymbol{x}$ given a fixed set of parameters $\boldsymbol{\phi}$, the likelihood refers to the value of $f$ as a function of $\boldsymbol{\phi}$ given a fixed value of $\boldsymbol{x}$.

Searches for \gls{bsm} physics are typically centred around the measurement of several disjoint binned distributions (called \textit{channels} $c$) that are each associated with different event selection criteria (as opposed to different scattering processes) yielding observed event counts $\boldsymbol{n}$. In such counting experiments where each event is independently drawn from the same underlying distribution, each bin is described by a Poisson term. The Poisson probability to observe $n$ events with a expectation of $\nu$ events, is given by
\begin{equation}
	\mathrm{Pois}(n\vert\nu) = \frac{\nu^n}{n!}e^{-\nu}.
\end{equation}
The expectation $\nu_i$ in each bin $i$ can be parametrised through the introduction of a signal strength parameter $\mu_{\mathrm{sig}}$ by
\begin{equation}
	\nu_i = \mu_{\mathrm{sig}}s_i + b_i,
\end{equation}
where $s_i$ and $b_i$ are the bin-wise expected signal and background rates, respectively. The signal strength $\mu_{\mathrm{sig}}$ and is used as \gls{poi} in fits to data. Fixing $\mu_{\mathrm{sig}} = 0$ yields a \gls{sm} expectation, while $\mu_{\mathrm{sig}} = 1$ represents a \textit{signal-plus-background} description at nominal signal cross section. Scanning multiple values of $\mu_{\mathrm{sig}}$ allows to set limits on the visible cross sections of the signal models considered in the search. 
  
For each of the channels, the total event rate is the sum over a set of physics processes, called \textit{samples}. The sample-wise event rates generally depend on the model parameters $\boldsymbol{\phi}$




\section{Parameter estimation}

\section{Statistical tests}