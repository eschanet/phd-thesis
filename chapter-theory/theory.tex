	%!TEX root = ../thesis.tex
%*******************************************************************************
%*********************************** Theory chapter *****************************
%*******************************************************************************

\chapter{Theory}

\ifpdf
    \graphicspath{{chapter-theory/Figs/Raster/}{chapter-theory/Figs/PDF/}{chapter-theory/Figs/}}
\else
    \graphicspath{{chapter-theory/Figs/Vector/}{chapter-theory/Figs/}}
\fi

This chapter outlines the basic principles 

\section{The Standard Model of Particle Physics}

\nomenclature{SM}{Standard Model of Particle Physics}
\nomenclature{LEP}{Large Electron Positron Collider}
\nomenclature{QED}{Quantum Electrodynamics}

By the end of the 1920s, quantum mechanics and general relativity had been relatively well established and the consensus among physicists was that matter was made of nuclear atoms consisting of electrons and protons. During the 1930s, a multitude of new experimental discoveries and theoretical puzzles excited physicists in three main fields of research: nuclear physics, cosmic rays and relativistic quantum mechanics. The following years and decades saw particle physics emerge as a result of these currents ultimately flowing together.

Since these early times of particle physics research, physicists have made extraordinary progress in describing nature at the subatomic scale. Today, a century later, the resulting theoretical framework, the Standard Model of Particle Physics (SM), is the most fundamental theory of nature to date. It provides an extremely precise description of the interactions of elementary particles and---using the Large Electron Positron collider (LEP)---has been tested and verified to an unprecedented level of accuracy up to the electroweak (EWK) scale. Given the unprecedented success of SM, it is not surprising that its history is paved with numerous awards for both experimental and theoretical work. In 1964, the Nobel prize was awarded to Feynman, Schwinger and Tomonoga for their fundamental work in quantum electrodynamics (QED). This quantum field theory allows to precisely calculate fundamental processes as e.g. the anomalous magnetic moment of the electron to a relative experimental uncertainty of $2.3 \times 10^{-10}$~\cite{Mohr:2015ccw}. In 1979, Glashow, Weinberg and Salam were awarded with the Nobel prize for their work towards electroweak unification. The most prominent recent progress is undoubtedly the discovery of the Higgs boson, not only resulting in the Nobel prize being awarded to Englert and Higgs, but also completing the SM, roughly 50 years after the existence of the Higgs boson had been theorised. 

	\todo[inline]{Couplings and masses are measured from experiment}
	
	The following theoretical descriptions and notations largely follow~\cite{Peskin:1995ev}. 


\subsection{Quantum field theories}

\nomenclature{QFT}{Quantum Field Theory}

Formally, the SM is a collection of quantum field theories. Quantum field theory (QFT) is similar to quantum mechanics in the sense that is is the application of 

\subsection{Renormalisation}

\subsection{Particle content}


\section{Supersymmetry}