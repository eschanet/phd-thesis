	%!TEX root = ../thesis.tex
%*******************************************************************************
%*********************************** Theory chapter *****************************
%*******************************************************************************

\chapter{Theory}

\ifpdf
    \graphicspath{{chapter-theory/Figs/Raster/}{chapter-theory/Figs/PDF/}{chapter-theory/Figs/}}
\else
    \graphicspath{{chapter-theory/Figs/Vector/}{chapter-theory/Figs/}}
\fi

This chapter starts with an outline of the basic principles and concepts of the Standard Model of Particle Physics (SM), the theoretical framework describing nature on the level of elementary particles. This is followed by an introduction to supersymmetry, a promising class of theories aiming to solve some of the shortcomings of the SM.

By no means intended to be a full description, this chapter merely tries to highlight the important relations and consequences of the SM and supersymmetry. The mathematical description of this chapter largely follows~\cite{Peskin:1995ev} for the SM and~\cite{Martin:1997ns} for supersymmetry.

\section{The Standard Model of Particle Physics}

\nomenclature[z-SM]{SM}{Standard Model of Particle Physics}
\nomenclature[z-LEP]{LEP}{Large Electron Positron Collider}
\nomenclature[z-QED]{QED}{Quantum Electrodynamics}

By the end of the 1920s, quantum mechanics and general relativity had been relatively well established and the consensus among physicists was that matter was made of nuclear atoms consisting of electrons and protons. During the 1930s, a multitude of new experimental discoveries and theoretical puzzles excited physicists in three main fields of research: nuclear physics, cosmic rays and relativistic quantum mechanics. The following years and decades saw particle physics emerge as a result of these currents ultimately flowing together.

Since these early times of particle physics research, physicists have made extraordinary progress in describing nature at the subatomic scale. Today, a century later, the resulting theoretical framework, the Standard Model of Particle Physics, is the most fundamental theory of nature to date. It provides an extremely precise description of the interactions of elementary particles and---using the Large Electron Positron collider (LEP)---has been tested and verified to an unprecedented level of accuracy up to the electroweak (EWK) scale. Given the unprecedented success of SM, it is not surprising that its history is paved with numerous awards for both experimental and theoretical work. In 1964, the Nobel prize was awarded to Feynman, Schwinger and Tomonoga for their fundamental work in quantum electrodynamics (QED). This quantum field theory allows to precisely calculate fundamental processes as e.g. the anomalous magnetic moment of the electron to a relative experimental uncertainty of $2.3 \times 10^{-10}$~\cite{Mohr:2015ccw}. In 1979, Glashow, Weinberg and Salam were awarded with the Nobel prize for their work towards electroweak unification. The most prominent recent progress is undoubtedly the discovery of the Higgs boson, not only resulting in the Nobel prize being awarded to Englert and Higgs, but also completing the SM, roughly 50 years after the existence of the Higgs boson had been theorised. 

	\improvement{Couplings and masses are measured from experiment}
		
\subsection{Particle content of the SM}

\begin{table}
	\centering
	\setlength\heavyrulewidth{0.2ex}
	\small
	\caption{Names, electric charges and masses (rounded to three significant digits if known to that precision) of all observed fermions in the SM~\cite{pdg}.}
	\begin{tabular} {c c c c c}
		
		\toprule
				& generation & particle & electric charge [$e$] & mass \\ 
		\midrule 
				\multirow{6}{*}{leptons}& \multirow{2}{*}{1} & electron ($e$)& $-1$ & \SI{511}{\keV}\\
				& & electron neutrino ($\nu_e$) & 0 & < \SI{2}{\eV} \\
				& \multirow{2}{*}{2} & muon ($\mu$)& $-1$ & \SI{106}{\MeV}\\
				& & muon neutrino ($\nu_\mu$) & 0 & < \SI{0.19}{\MeV} \\
				& \multirow{2}{*}{3} & tau ($\tau$)& $-1$ & \SI{1.78}{\GeV}\\
				& & tau neutrino ($\nu_\tau$) & 0 & < \SI{18.2}{\MeV} \\
		\midrule 
				\multirow{6}{*}{quarks}& \multirow{2}{*}{1} & up ($u$)& $\frac{2}{3}$ & \SI{2.3}{\MeV}\\
				& & down ($d$) & $-\frac{1}{3}$ & \SI{4.8}{\MeV} \\
				& \multirow{2}{*}{2} & charm ($c$)& $\frac{2}{3}$ & \SI{1.28}{\GeV}\\
				& & strange ($s$) & $-\frac{1}{3}$ &\SI{95}{\MeV} \\
				& \multirow{2}{*}{3} & top ($t$)& $\frac{2}{3}$ & \SI{173}{\GeV}\\
				& & bottom ($b$) & $-\frac{1}{3}$ & \SI{4.18}{\GeV} \\
		\bottomrule
	\end{tabular}\vspace{3mm}
	\label{tab:particles_fermions}   
\end{table}

The SM successfully describes \change{Neutrino masses not in SM!} ordinary matter as well as their interactions, namely the electromagnetic, weak and strong interactions. Gravity is the only fundamental force not described within the SM. The particles in the SM are classified into two main categories, depending on their spin. Particles with half-integer spin follow Fermi-Dirac statistics and are called fermions. As they are subject to the Pauli exclusion principle, they make up ordinary matter. Particles with integer spin follow Bose-Einstein statistics and mediate the fundamental interactions between fermions. 

%\subsubsection*{Fermions} 

Fermions are further divided into leptons and quarks, which each come in three generations with increasing masses\footnote{Neutrinos might not exist in a normal mass hierarchy but could also have an inverted mass hierarchy.\info{Need ref}}. The three electrically charged leptons are each associated with a corresponding neutral neutrino (more on this \textit{association} in chapter\info{need ref}). While the SM assumes massless neutrinos, the observation of neutrino oscillations~\cite{Fukuda:1998mi} implies the existence of at least two massive neutrinos. By extending the SM to allow non-vanishing neutrino masses, neutrino oscillations can be introduced through lepton generation mixing, described by the Pontecorvo-Maki-Nakagawa-Sakata (PMNS)\nomenclature[z-PMNS]{PMNS}{Pontecorvo–Maki–Nakagawa–Sakata} matrix~\cite{PMNS:1962mu}. Apart from an electric charge, the six quarks also carry a colour charge. There are three types of colour charge: \textit{red}, \textit{green} and \textit{blue} as well as their respective anti-colours. The mixing in the quark sector through the weak interaction can be described by the Cabibbo-Kobayashi-Maskawa (CKM)\nomenclature[z-CKM]{CKM}{Cabibbo-Kobayashi-Maskawa} matrix~\cite{CKM:1973fv}. Finally, each fermion comes with its own anti-particle with same mass and spin, but inverted charge-like quantum numbers\footnote{The exact nature of anti-neutrinos is still an open question and ties into whether or not the neutrino mass matrix contains non-vanishing Majorana mass terms.}. All fermions in the SM are listed in \cref{tab:particles_fermions}.

%\subsubsection*{Bosons}

\begin{table}
	\centering
	\setlength\heavyrulewidth{0.2ex}
	\small
	\caption{Names, electric charges and masses (rounded to three significant digits if known to that precision) of all observed bosons in the SM~\cite{pdg}.}
	\begin{tabular} {c c c c}
	\toprule
		particle & spin & electric charge [$e$]& mass \\ 
	\midrule
		photon ($\gamma$) & 1 & 0 & 0\\
		gluon ($g$) & 1 & 0 & 0 \\
		$W^\pm$ & 1 & $\pm 1$ & \SI{80.4}{\GeV} \\
		$Z^0$ & 1 & 0 & \SI{91.2}{\GeV} \\
		Higgs boson ($H$) & 0 & 0 & \SI{125}{\GeV} \\
	\bottomrule					
	\end{tabular}\vspace{3mm}
	\label{tab:particles_bosons}   
\end{table}

The fundamental forces described by the SM are propagated by bosons with spin $1\hbar$. The photon $\gamma$ couples to electrically charged particles and mediates the electromagnetic interaction. As the photon is massless, the electromagnetic force has infinite range. The strong force is mediated by gluons carrying one unit of colour and one unit of anti-colour. Due to colour-confinement, colour charged particles like quarks and gluons cannot exist as free particles and instead will always form colour-neutral bound states. Although nine gluon states would theoretically be possible, only eight of them are realised in nature: the colour-singlet state $\frac{1}{\sqrt{3}}(\ket{r\bar{r}}+\ket{g\bar{g}}+\ket{b\bar{b}})$ would be colour-neutral result in long-range strong interactions, which have not been observed.\info{Might want to explain this later once I introduced the gauge groups?} Finally, the weak force is mediated by a total of three bosons, two charged $W$-bosons $W^+$ and $W^-$, and a neutral $Z$-boson. The mediators of the weak force are massive, resulting in a finitely ranged interaction. The $W^\pm$ and $Z$ bosons gain their masses through the Higgs mechanism (discussed in chapter \info{need ref}), resulting in a massive spin-0 boson, called the Higgs boson. All bosons known to the SM are listed in \cref{tab:particles_bosons}.


\subsection{The SM as a gauge theory}\label{ch:gauge_theory}

\nomenclature[z-QFT]{QFT}{Quantum Field Theory}
\nomenclature[z-QCD]{QCD}{Quantum Chromodynamics}

Formally, the SM is a collection of a special type of quantum field theories, called gauge theories. Quantum field theory (QFT) is the application of quantum mechanics to dynamical systems of fields, just as quantum mechanics is the quantisation of dynamical systems of particles. QFT provides a uniform description of quantum mechanical particles and classical fields, while including special relativity.

In classical mechanics, the fundamental quantity  is the action $S$, which is the time integral of the Lagrangian $L$, a functional characterising the state of a system of particles in terms of generalised coordinates $q_1, \dots, q_n$. In field theory, the Lagrangian can be written as spatial integral of a Lagrangian density $\Lagr(\phi_i,\partial_\mu\phi_i)$, that is a function of one or more fields $\phi_i$ and their spacetime derivates $\partial_\mu\phi_i$. For the action, this yields
\begin{equation}
	S = \int L\diff t = \int\Lagr\left(\phi_i,\partial_\mu\phi_i\right)\Diff4 x.
\end{equation}
In the following, the Lagrangian density $\Lagr$ will simply be referred to as the \textit{Lagrangian}.

Using the principle of least action $\delta S = 0$, the equation of motions for each field are given by the Euler-Lagrange-equation,
\begin{equation}
	\partial_\mu\left(\frac{\partial\Lagr}{\partial\left(\partial_\mu\phi_i\right)}\right)-\frac{\partial\Lagr}{\partial\phi_i}=0.
	\label{eq:euler_lagrange}
\end{equation}
As opposed to the Hamiltonian formalism, the Lagrange formulation of field theory is especially well suited in this context, as it exhibits explicit Lorentz-invariance. This is a direct consequence of the principle of least action, since boosted extrema in the action will still be extrema for Lorentz-invariant Lagrangians.

\improvement{Explicitly derive the Euler-Lagrange equations? Cf. Peskins Ch.2.2.}

Symmetries are of central importance in the SM. As Emmy Noether has famously shown in 1918~\cite{physics/0503066} for classical mechanics, every continuous symmetry of the action has a corresponding conservation law. In the context of classical field theory, each generator of a continuous internal or spacetime symmetry transformation leads to a conserved current, and thus to a conserved charge. In QFTs, quantum versions of Noether's theorem, called Ward–Takahashi identities~\cite{PhysRev.78.182,Takahashi1957} for Abelian theories and Slavnov–Taylor identities~\cite{THOOFT1971173,TAYLOR1971436,Slavnov1972} for non-Abelian theories relate the conservation of quantum currents and charge-like quantum numbers to continuous global symmetries of the Lagrangian.\unsure{Check correctness of formulation}

From a theoretical point of view, the SM can be described by a non-Abelian Yang-Mills type \improvement{cite YM} gauge theory based on the symmetry group
\begin{equation*}
	SU(3)_C \otimes SU(2)_L \otimes U(1)_Y,
\end{equation*}
where $U(n)$ ($SU(n)$) describes (special) unitary groups, \ie the Lie groups of $n\times n$ unitary matrices (with determinant 1, if special). $SU(3)_C$ generates quantum chromodynamics (QCD), \ie the interaction of particles with colour charge through exchange of gluons, and $SU(2)_L \otimes U(1)_Y$ generates the electroweak interaction. Here, the subscript $Y$ represents the weak hypercharge, while the $L$ indicates that $SU(2)_L$ only couples to left-handed particles (right-handed antiparticles).


\subsubsection{Gauge principle}

The gauge principle is fundamental to the SM and dictates that the existence of gauge fields is directly related to symmetries under local gauge transformations. QED, being the simplest gauge theory, can be taken to illustrate this important principle. The free Dirac Lagrangian for a single, non-interacting fermion with mass $m$ is given by
\begin{equation}
	\Lagr_\mathrm{Dirac}=\bar{\psi}\left(i\gamma^\mu\partial_\mu - m\right)\psi,
	\label{eq:dirac_lagrangian}
\end{equation}
where $\psi$ is a four-component complex spinor field, $\bar{\psi} = \psi^\dagger\gamma^0$, and $\gamma^\mu$ with $\mu = 0,1,2,3 $ are the Dirac matrices with the usual anticommutation relations generating a matrix representation of the Dirac algebra 
\begin{equation}
	\{\gamma^\mu,\gamma^\nu\} \equiv \gamma^\mu\gamma^\nu + \gamma^\nu\gamma^\mu = 2\eta^{\mu\nu}\mathbb{1}_4.
\end{equation}

It is worth noting that the free Dirac Lagrangian is invariant under a global $U(1)$ transformation
\begin{equation}
	\psi \rightarrow e^{i\theta}\psi,
\end{equation}
where the phase $\theta$ is spacetime independent and real. In order to produce the physics of electromagnetism, the free Dirac Lagrangian however has to be invariant under \textit{local} $U(1)$ phase transformations, which is not the case, as the transformed Lagrangian picks up an additional term from the spacetime derivative of the phase $\partial_\mu\theta(x)$.

In order for the Dirac Lagrangian to become invariant under a local gauge transformation, a new vector field $A_\mu(x)$ has to be introduced and the partial derivative has to be replaced with the covariant derivative\footnote{The prescription of achieving local gauge invariance by replacing $\partial_\mu$ with $D_\mu$ is called \textit{minimal coupling}.}
\begin{equation}
	\partial_\mu \rightarrow \codiff_\mu \equiv \partial_\mu + ieA_\mu,
\end{equation}
where $e$ is the coupling of the fermion field to the gauge field $A_\mu$ and can be identified with the elementary charge. This leads to a Lagrangian that is invariant under the transformations
\begin{equation}
	\psi \rightarrow e^{i\theta\left(x\right)}\psi,\, \, \, \, \, \, \, \, \, \, A_\mu \rightarrow A_\mu - \frac{1}{e}\partial_\mu\theta(x).
	\label{eq:gauge_field}
\end{equation}

The modified Lagrangian now includes a term for interactions between the gauge field and the fermion field
\begin{equation}
\begin{split}
	\Lagr &= \Lagr_\mathrm{Dirac} + \Lagr_\mathrm{int} \\
		&= \bar{\psi}\left(i\gamma^\mu\partial_\mu - m\right)\psi - \left(e\bar{\psi}\gamma^\mu\psi\right)A_\mu,
	\label{eq:modified_lagrangian}
\end{split}
\end{equation}
and is indeed invariant under a local phase transformation. Yet, it still cannot be complete as it is missing a free kinetic term for the gauge field $A_\mu$ itself. For a vector field, the free term is described by the Proca Lagrangian
\begin{equation}
	\Lagr_\mathrm{Proca} = -\frac{1}{4}F_{\mu\nu}F^{\mu\nu} + \frac{1}{2}m_A^2A^\nu A_\nu,
\end{equation}
where $F^{\mu\nu}\equiv\left(\partial^\mu A^\nu-\partial^\nu A^\mu\right)$ is the field strength tensor that is invariant under the transformation in \cref{eq:gauge_field}. Since $A^\nu A_\nu$ is not invariant under the same transformation, the only way to keep the full Lagrangian invariant under a local phase transformation is by requiring $m_A=0$, \ie the introduced gauge field $A_\mu$ has to be massless, giving the Maxwell Lagrangian (ultimately generating the Maxwell equations)
\begin{equation}
	\Lagr_\mathrm{Maxwell} = -\frac{1}{4}F_{\mu\nu}F^{\mu\nu}.
\end{equation}

This finally yields the full Lagrangian
\begin{equation}
\begin{split}
		\Lagr_\mathrm{QED} & = \Lagr_\mathrm{Dirac} + \Lagr_\mathrm{Maxwell} + \Lagr_\mathrm{int} \\
	  				& = \bar{\psi}\left(i\gamma^\mu\partial_\mu\right)\psi - m\bar{\psi}\psi - \frac{1}{4}F^{\mu\nu}F_{\mu\nu} - \left(e\bar{\psi}\gamma^\mu\psi\right)A_\mu \\
					& = \bar{\psi}\left(i\gamma^\mu\codiff_\mu - m\right)\psi - \frac{1}{4}F^{\mu\nu}F_{\mu\nu},
\end{split}
\end{equation}
which can be identified to be the full Lagrangian of QED. The introduced gauge field $A_\mu$ is therefore nothing else but the electromagnetic potential with its associated massless particle, the photon. Thus, by applying the gauge principle on the free Dirac Lagrangian, \ie forcing a global phase invariance to hold locally, a new massless gauge field including interaction terms with the existing fields in the Lagrangian has to be introduced. In the case of the free Dirac Lagrangian, local gauge invariance produces all of quantum electrodynamics.

As the phase transformation in \cref{eq:gauge_field} is part of the unitary group $U(1)$, this symmetry is called $U(1)$ \textit{gauge symmetry}. 

\subsubsection{Quantum chromodynamics}

Formally, the strong interaction between quarks and gluons in the SM is described by quantum chromodynamics (QCD), which is based on the $SU(3)_C$ gauge group, where the subscript $C$ refers to colour. As Yang and Mills have shown in 1954 \cite{PhysRev.96.191}, requiring a global phase invariance to hold locally is perfectly possible in the case of any continuous symmetry group. In the case of a general non-Abelian symmetry group, a gauge-invariant Lagrangian can be constructed in a similar manner as previously in the case of $U(1)$.

Considering the general case of $SU(n)$, represented by a set of $n\times n$ unitary matrices, then a total of $n$ fermion fields in an $n$-dimensional multiplet $\Psi = (\psi_1,\dots,psi_n)^T$ are needed. The covariant derivative associated with the general transformation of the fields is given by 
\begin{equation}
	\codiff_\mu = \partial_\mu - igA\mu^a t^a \quad a = 1,\dots,(n^2-1)
\end{equation}
where $t^a$ are the $n^2-1$ generators of $SU(n)$ 


\subsection{Renormalisation and divergencies}


 

\section{Supersymmetry}