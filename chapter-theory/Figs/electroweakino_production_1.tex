\documentclass[class=minimal,border={10pt 16pt 10pt 16pt}]{standalone}
\usepackage[force]{feynmp-auto}
\usepackage{xcolor}
	\usepackage{amsmath}
	\usepackage{fontspec}
	\usepackage{unicode-math}
	\usepackage{microtype}

	\setmainfont{Libertinus Serif}
	\setsansfont{Libertinus Sans}
	\setmathfont{Libertinus Math}
\begin{document}
	\unitlength=0.65mm
	\begin{fmffile}{electroweak_production_1_1}
		  \begin{fmfgraph*}(50,30)
		    % Define two vertices on the left, but only `i2' will be actually used.
		    \fmfleft{i1,i2}
		    % The same on the right.
		    \fmfright{o1,o2}
		    % Define the vertex for the blob.
		    \fmf{fermion,label=$$,label.side=left,tension=1.0}{i1,v1}
		    \fmf{fermion,label=$$,label.side=left,tension=1.0}{v1,i2}

		    \fmf{boson,label=$W^\pm$,label.side=left,tension=1.0}{v1,v2}

		    \fmf{plain,label=$$,label.side=left,tension=1,foreground=red}{o1,v2}
		    \fmf{plain,label=$$,label.side=left,tension=1,foreground=red}{o2,v2}
		    
			\fmffreeze
		    \fmf{boson,foreground=red}{o1,v2}
		    \fmf{boson,foreground=red}{o2,v2}
		    
		    \fmflabel{$q$}{i1}
		    \fmflabel{$\bar{q}$}{i2}
		    
		    \fmflabel{\textcolor{red}{$\tilde{\chi}_2^0$}}{o1}
		    \fmflabel{\textcolor{red}{$\tilde{\chi}_1^\pm$}}{o2}

		  \end{fmfgraph*}
	\end{fmffile}
\end{document}                    