\documentclass[class=minimal,border={10pt 15pt 10pt 15pt}]{standalone}
\usepackage[force]{feynmp-auto}
\usepackage{xcolor}
	\usepackage{amsmath}
	\usepackage{fontspec}
	\usepackage{unicode-math}
	\usepackage{microtype}

	\setmainfont{Libertinus Serif}
	\setsansfont{Libertinus Sans}
	\setmathfont{Libertinus Math}
\begin{document}
	\unitlength=0.65mm
	\begin{fmffile}{qed_anomalous_moment_1}
		  \begin{fmfgraph*}(70,40)
		    % Define two vertices on the left, but only `i2' will be actually used.
		    \fmfleft{i1,i2}
		    % The same on the right.
		    \fmfright{o1,o2}
		    % Define the vertex for the blob.
		    \fmftop{b}
		    \fmf{fermion,label=$$,label.side=left,tension=1.0}{i1,v1}
		    \fmf{fermion,label=$$,label.side=left,tension=1.0}{v2,o1}
		    \fmf{photon, tension=1}{v3,b} 
%		    \fmfblob{.15w}{b}
		    \fmf{fermion,label=$$,label.side=left,tension=0.45}{v1,v3}
		    \fmf{fermion,label=$$,label.side=left,tension=0.45}{v3,v2}
		    \fmf{photon, label=$\gamma$,tension=0.45}{v1,v2} 
		    % Labels on vertices.
		    \fmflabel{$\mu$}{i1}
		    \fmflabel{$\mu$}{o1}
		  	\fmflabel{$\gamma$}{b}
		  \end{fmfgraph*}
	\end{fmffile}
\end{document}