% ************************** Thesis Acknowledgements **************************

\begin{acknowledgements}[Acknowledgements]  

Finally, this is where I want to thank everyone that has made this work possible. Most and foremost, I would like to thank Prof.\@\xspace Dr.\@\xspace Dorothee Schaile for giving me the opportunity to work at the chair of elementary particle physics, for allowing me to attend a number of workshops and conferences, and, especially, for always having an open door in case I needed some advice.

I am also deeply grateful to PD\@\xspace Dr.\@\xspace Jeanette Lorenz, not only for the excellent supervision all through my Bachelor's, Master's and PhD theses, but especially also for the many stimulating discussions and words of advice, as well as the numerous constructive comments regarding this thesis. Your guidance and feedback during the last couple of years have been invaluable. Thank you for continuously encouraging me to actively help shaping efforts to search for Supersymmetry in ATLAS. Thank you also for your relentless and fruitful efforts to facilitate a regular exchange of ideas, even during times of pandemic-enforced home-office and isolation.

I would further like to express my gratitude towards Prof.\@\xspace Dr.\@\xspace Wolfgang D\"unnweber for agreeing to provide the second review of this thesis, especially since it got much longer than initially anticipated. I am further grateful to Prof.\@\xspace Dr.\@\xspace Andreas Burkert, Prof.\@\xspace Dr.\@\xspace Gerhard Buchalla, Prof.\@\xspace Dr.\@\xspace Thomas Kuhr and Prof.\@\xspace Dr.\@\xspace Thomas Preibisch for agreeing to take part in the examination commission.

Many thanks to all the members of the \onelepton analysis team as well as the numerous colleagues involved in the pMSSM efforts in ATLAS. Most of the results presented herein are a product of a collaborative effort. Furthermore, many of the discussions we had have actively shaped this thesis.

I would like to acknowledge the support of the Luxembourg National Research Fund (FNR) for funding my PhD project, especially because high-energy physics research is still virtually inexistent in Luxembourg. Although a founding member of numerous European endeavours, Luxembourg is still one of only three European countries that are not a member of CERN, a circumstance that I regard as deeply disappointing and regretful. My sincere hope is that the future sees a growing high-energy physics community in Luxembourg. Funding single, external projects like mine certainly is an important step to allow this to happen.

Furthermore, I would like to thank all the members of the institute for always ensuring a warm and friendly atmosphere, for the numerous work-unrelated activities, for the countless, interesting discussions, and for bearing with me when the shared group disks were running out of space again.
Special thanks go to Dr.\@\xspace Nikolai Hartmann for being an awesome office partner, for the countless discussions on the sense and nonsense of data analysis with and without ATLAS software, and for sharing with me your passion for coffee and for obscure (and sometimes esoteric) programming languages and software packages.
Special thanks also go to Dr.\@\xspace Ferdinand Krieter (I agree, the title does look weird) and Paola Arrubarrena for the many non-physics chats, for sharing my somewhat unusual sense of humour, and for enduring my virtual rants during the writing phase in home-office.
Many thanks also to Dr.\@\xspace Michael Holzbock, Dr.\@\xspace Andrea Matic and David Koch for patiently playing the receiving end during many of my rubber duck debugging sessions. 

My gratitude also goes to Yannick Erpelding and Nick Beffort. I am deeply grateful for the many years of friendship and for the myriad of memorable moments we have lived through so far. Thank you for being awesome friends and for looking out for me.

Finally, and most importantly, I owe my deepest gratitude to my family, in particular to my parents and to my sister for always supporting me and encouraging me to expand my horizons, but also to my wonderful partner Nathalie M\"unster for enduring the numerous inevitable meltdowns during writing up in times of a pandemic, for being an endless source of stress relief, and, most notably, for always being on my side and making me laugh every single day. To Pandu I want to say: stay vigilant, my good boy, for much more adventurous and treat-filled times are ahead.  

%\vspace{5em}
%
%\begin{tikzpicture}[remember picture,overlay,shift={(current page text area.south east)}]
%\node [above right]{\parbox{0.70\textwidth}{
%\raggedright\textit{``I may not have gone where I intended to go, but I think I have ended up where I needed to be.''
%}
%\par\raggedleft--- \textup{Douglas Adams, The Long Dark Tea-Time of the Soul}
%\vspace{10em}
%}};
%\end{tikzpicture}
%
%


\end{acknowledgements}
