%!TEX root = ../thesis.tex
%*******************************************************************************
%*********************************** Signal region optimisation *********
%*******************************************************************************


\chapter{Signal region optimisation}

\ifpdf
    \graphicspath{{chapter-optimisation/Figs/Raster/}{chapter-optimisation/Figs/PDF/}{chapter-optimisation/Figs/}}
\else
    \graphicspath{{chapter-optimisation/Figs/Vector/}{chapter-optimisation/Figs/}}
\fi

\glsreset{sr}

In order to discover the rare signals predicted by the \gls{susy} models considered, dedicated kinematic regions enriched in signal events, so called \gls{sr} are constructed. They are optimised to be able to discover a maximum number of the signal models considered in the analysis. In this chapter, the \gls{sr} optimisation procedures leading to the final \glspl{sr} are introduced and discussed. 

\section{Optimisation methods}

All optimisation methods used in the following need a figure of merit that should be maximised in order to define the best performing setup. While the multidimensional cut scan in \cref{sec:n-dim-scan} and the N-1 plots approach in \cref{sec:n-1-scan} use the binomial discovery significance $Z_\mathrm{B}$ introduced in~\cref{sec:sensitivity_estimation}, the fit scan procedure in \cref{sec:fit-scan} aims to maximise the area of the expected exclusion contour.

\subsection{Multidimensional cut scan}\label{sec:n-dim-scan}

\subsection{N-1 plots}\label{sec:n-1-scan}

\subsection{Fit scans}\label{sec:fit-scan}


