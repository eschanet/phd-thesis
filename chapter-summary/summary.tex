%!TEX root = ../thesis.tex
%*******************************************************************************
%*********************************** Conclusions *********
%*******************************************************************************

%\titleformat{\chapter}[display]
%	{\normalfont\LARGE}
%	{\filleft\MakeUppercase{\chaptertitlename}\hspace{0.5cm}\rlap{\resizebox{!}{1.5cm}{\thechapter}~\rule{5cm}{1.5cm}}\hspace{0.5cm}}
%  	{10pt}
%  	{\bf\LARGE\filright}
%\titlespacing*{\chapter}{0pt}{30pt}{20pt}

\chapter{Conclusions}

\ifpdf
    \graphicspath{{chapter-summary/Figs/Raster/}{chapter-summary/Figs/PDF/}{chapter-summary/Figs/}}
\else
    \graphicspath{{chapter-summary/Figs/Vector/}{chapter-summary/Figs/}}
\fi


This thesis presented a search for direct production of electroweakinos in events with one lepton, missing transverse momentum and a Higgs boson decaying into two \textit{b}-jets. The full dataset \gls{lhc} Run~2 dataset of \onethirtynineifb of $pp$ collisions recorded at $\sqrt{s} = \SI{13}{\TeV}$ with the ATLAS detector was analysed. The search targets a simplified $\charg\neutr$ pair-production model with subsequent decays into $W$ and Higgs bosons together with two $\lsp$. Branching fractions of 100\% are assumed in each branch. While the two $\lsp$ completely escape the detector, resulting in significant amounts of missing transverse momentum that can be triggered on, the search targets a $W$ boson decay into a lepton--neutrino pair and a Higgs boson decay into a \textit{b}-jet pair.

The \textit{b}-jet pair offers a powerful discriminative handle as its invariant mass, $\mbb$, shows a characteristic peak around the Higgs boson mass. In order to achieve sensitivity to a maximum variety of kinematic regimes, nine search regions are defined, optimised using a dedicated optimisation procedure. All search regions are situated on the Higgs boson mass peak and are designed to be mutually exclusive through their requirements on the transverse mass, $\mt$, and the contransverse mass, $\mct$. A single likelihood is constructed, statistically combining all search regions into a two-dimensional shape-fit that exploits the varying shapes of the \gls{susy} signal and \gls{sm} background processes. Contributions from \gls{sm} background processes in the search regions are estimated either using dedicated control regions and transfer factors, or directly from \gls{mc} simulation normalised to their theoretical cross section. A combined likelihood containing terms for all control and signal regions including all systematic uncertainties considered was constructed and fit to data. 

No significant excess was found in any of the search regions, and hence limits on the model parameters are set. Due to the introduction of the two-dimensional shape-fit and the unprecedented amount of \onethirtynineifb of data analysed, the limits set by previous searches targeting the same simplified model can be significantly extended. For a light \gls{lsp}, $\charg/\neutr$ masses up to $\SI{740}{\GeV}$ can be excluded at 95\% CL. In the case of a heavier \gls{lsp} with $m(\lsp)\approx\SI{250}{\GeV}$, the limits on the $\charg/\neutr$ masses weaken to about $\SI{600}{\GeV}$. At the time of writing, the limits obtained by this search represent the most stringent constraints on $\charg\neutr$ pair-production set by ATLAS in the context of the simplified model considered~\cite{ATL-PHYS-PUB-2020-020}.

The absence of physics beyond the Standard Model in the full Run~2 dataset of the \gls{lhc} in the search presented herein, is in line with the results of other \gls{susy} searches performed by ATLAS. The existence of gluinos and squarks at the $\SI{}{\TeV}$-scale was already severely challenged by the end of Run~1 of the \gls{lhc}. Due to the large integrated luminosity available through the full Run~2 dataset and the improved analysis techniques and strategies developed over the last years, the typically weaker limits on electroweakinos and sleptons are also significantly increasing and in some cases approach the $\SI{1}{\TeV}$ mark. 
%The diverse \gls{susy} search programs at ATLAS and CMS thus heavily constrain the existence of \gls{susy} at the $\SI{}{\TeV}$ scale.

Given these constraints, one might be tempted to discard the existence of \gls{susy} at the \gls{lhc} altogether. Such conclusions would, however, be drawn much too early. On the one hand, only a fraction of the total integrated luminosity the \gls{lhc} is designed to deliver is available. By the end of the its lifetime (including the high-luminosity upgrade), a projected amount of $\SI{3000}{\per\femto\barn}$~\cite{Apollinari:2116337} will have been delivered to the particle physics experiments by the \gls{lhc}. A multitude of supersymmetric models not accessible with the full Run~2 dataset using today's analyses will only come into reach in the coming years of the \gls{lhc}.
On the other hand, and more importantly in the context of this thesis, most limits derived by \gls{susy} searches assume specific simplified models and are thus only valid in the context of models satisfying the respective simplified model assumptions. In any realistic \gls{susy} scenario, assumptions like 100\% branching fractions or only a small set of supersymmetric particles not kinematically decoupled are most likely not exactly fulfilled. Thus, the quoted simplified model limits can in general not be trivially interpreted as the true underlying constraint on the respective parameter of a more realistic \gls{susy} scenario. The true constraints on supersymmetric model parameters will in general be weaker than the limits frequently quoted.
 
Due to the rapidly changing landscape of models for physics beyond the \gls{sm}, and the limited scope of parameter limits quoted by the experiments, re-interpretations of searches for supersymmetry are highly desirable and see significant interest from both the experimental as well as theory communities. With this in mind, the search for \gls{susy} presented herein was implemented to be fully re-usable and re-interpretable in the light of new \gls{bsm} models. This is achieved using a cyber-infrastructure called \textsc{Recast}~\cite{RECAST_cranmer}, relying on containerised workflows orchestrating parametrised job templates. Additionally, the full likelihood of the search was made publicly available in \texttt{JSON} format, allowing it to be incorporated in re-interpretation efforts like \textsc{SModelS}~\cite{SModelS1:2013mwa,SModelS2:2017neo} and \textsc{MadAnalysis5}~\cite{Goodsell:2020ddr,Fuks:2021wpe}. 
 
Large-scale re-interpretations using high-dimensional parameter spaces are especially interesting as they include complex \gls{susy} models producing more realistic scenarios than the usual simplified models. Such re-interpretations are, however, computationally extremely challenging and require approximations to make them computationally feasible. In this thesis, a method to generically approximate the full likelihoods of \gls{susy} searches was introduced and validated using a selection of ATLAS \gls{susy} searches. The search previously presented was then re-interpreted in the \gls{pmssm}, a 19-dimensional parameter space containing realistic \gls{susy} scenarios. Due to the assumption of 100\% branching fractions not being satisfied in many more realistic \gls{susy} scenarios, the sensitivity of the \onelepton search was found to be significantly reduced but a small fraction of models could still be excluded. The impact of the \onelepton search on electroweakino masses was investigated, revealing some sensitivity to $\tilde{\chi}_2^\pm\neutr$ production in addition to sensitivity towards models phenomenologically close to the simplified model initially considered. The impact of the \onelepton search on the \gls{dm} relic density was also investigated. While no conclusive statement could be made for models with a bino-like $\lsp$ due to the limited number of such models sampled in the relevant parameter space, some models with a wino-like $\lsp$ with the right relic density could be excluded. 
 
 Although hopes of quickly finding supersymmetric particles with the \gls{lhc} have not materialised, there is still a possibility of finding hints for physics beyond the \gls{sm} with \gls{lhc} experiments. Considerable regions of the parameter space of realistic \gls{susy} scenarios are still largely unconstrained and offer ample space for \gls{susy} to hide in. In order to provide a comprehensive overview of the constrained parameter space, it is crucial to design searches for Supersymmetry to be systematically re-interpretable, especially in light of complete and realistic \gls{susy} scenarios. Searches for \gls{bsm} physics are the tools that shine a light on the otherwise dark landscape that are the parameter spaces of \gls{bsm} theories. Allowing these tools to be re-usable significantly increases the area of parameter space they can shine a light onto.    
 