%!TEX root = ../thesis.tex
%*******************************************************************************
%*********************************** Conclusions *********
%*******************************************************************************

%\titleformat{\chapter}[display]
%	{\normalfont\LARGE}
%	{\filleft\MakeUppercase{\chaptertitlename}\hspace{0.5cm}\rlap{\resizebox{!}{1.5cm}{\thechapter}~\rule{5cm}{1.5cm}}\hspace{0.5cm}}
%  	{10pt}
%  	{\bf\LARGE\filright}
%\titlespacing*{\chapter}{0pt}{30pt}{20pt}

\chapter{Summary and Outlook}

\graphicspath{{chapter-summary/Figs/Vector/}{chapter-summary/Figs/}}

This thesis presented a search for direct production of electroweakinos in events with one lepton, missing transverse momentum and a Higgs boson decaying into two \textit{b}-jets.
The full dataset of Run~2 of the \gls{lhc}, amounting to \onethirtynineifb of $pp$ collisions at $\sqrt{s} = \SI{13}{\TeV}$, recorded with the ATLAS experiment, was analysed.
The search targets a simplified electroweakino ($\charg\neutr$) pair-production model with subsequent decays into $W$ and Higgs bosons together with two lightest neutralinos ($\lsp$). The $\lsp$ is the \gls{lsp} of the model, is electrically neutral and stable, and thus could be a good candidate for \gls{dm}.
It escapes the detector without leaving a trace, resulting, in general, in a significant amount of missing transverse momentum that can be triggered on.
The search further targets a $W$ boson decay into a lepton--neutrino pair and a Higgs boson decay into a \textit{b}-jet pair.

Both the lepton--neutrino and the \textit{b}-jet pairs offer powerful discriminative handles, exploited through the use of a number of kinematic observables in order to maximise the signal-to-background ratio in the phase space targeted.
Using a dedicated optimisation procedure, two sets of signal regions are defined, one targeting generic \gls{bsm} scenarios (called \textit{discovery} signal regions), and one optimised for the simplified model in question (called \textit{exclusion} signal regions). 
The exclusion signal regions are designed to be mutually exclusive through their requirements on the transverse mass ($\mt$) and the contransverse mass ($\mct$).
Contributions from \gls{sm} background processes in the signal regions originate primarily from $\ttbar$ and single top production, as well as $\wjets$ processes. Contributions from \gls{sm} backgrounds are estimated either with a semi-data-driven technique using dedicated control regions, or directly from \gls{mc} simulation.
A binned likelihood is constructed, statistically combining all exclusion signal regions into a two-dimensional shape-fit that exploits the varying shapes of the $\mt$ and $\mct$ distributions of \gls{susy} signal and \gls{sm} background processes. This approach allows to achieve sensitivity to a wide variety of kinematic regimes.
%A combined likelihood containing terms for all control and signal regions including all systematic uncertainties considered was constructed and fit to data. 

No significant excess has been observed in any of the signal regions, and thus model-dependent exclusion limits and model-independent upper limits on the visible cross section of \gls{bsm} processes have been derived.
Due to the introduction of the two-dimensional shape-fit and the unprecedented amount of \onethirtynineifb of \textit{pp} collision data analysed, the model-dependent exclusion limits set by previous searches targeting the same simplified model can be significantly extended.
For a massless \gls{lsp}, $\charg/\neutr$ masses up to $\SI{740}{\GeV}$ can be excluded at 95\% CL. In the case of a heavier \gls{lsp} with $m(\lsp)\approx\SI{250}{\GeV}$, the limits on the $\charg/\neutr$ masses weaken to about $\SI{600}{\GeV}$.
At the time of writing, the limits obtained by this search represent the most stringent constraints on $\charg\neutr$ pair-production set by ATLAS in the context of the simplified model considered~\cite{ATL-PHYS-PUB-2020-020}.
The model-independent 95\% CL upper limits on the visible cross section of \gls{bsm} processes vary between $\SI{0.26}{\femto\barn}$ and $\SI{0.11}{\femto\barn}$, depending on the signal region considered. 

The absence of physics beyond the \gls{sm} in the Run~2 dataset of the \gls{lhc} in the search presented herein, is in line with the results of other \gls{susy} searches performed by ATLAS and CMS.
While the existence of gluinos and squarks at the $\SI{}{\TeV}$-scale was already severely challenged by the end of Run~1 of the \gls{lhc}, the limits on electroweakinos and sleptons were, in general, weaker because of their smaller production cross sections.
Due to the large integrated luminosity available through the Run~2 dataset, and the improved analysis techniques and strategies developed over the last years, the limits on electroweakinos and sleptons are also significantly increasing, and in some cases start to approach the $\SI{1}{\TeV}$ mark~\cite{ATL-PHYS-PUB-2020-020,SUSY-2018-32}. 
%The diverse \gls{susy} search programs at ATLAS and CMS thus heavily constrain the existence of \gls{susy} at the $\SI{}{\TeV}$ scale.

Given these constraints, it might be tempting to discard the existence of \gls{susy} at the \gls{lhc} altogether. Such conclusions would, however, be drawn much too early.
On the one hand, \onethirtynineifb of \textit{pp} collision data only corresponds to a fraction of the total integrated luminosity the \gls{lhc} is designed to deliver. By the end of the lifetime of the high-luminosity \gls{lhc} upgrade, a projected amount of $\SI{3000}{\per\femto\barn}$~\cite{Apollinari:2116337} will have been delivered to the particle physics experiments.
Many supersymmetric models not accessible with the Run~2 dataset using today's analyses, will hence only come into reach in the upcoming runs of the \gls{lhc}.
On the other hand, most limits derived by \gls{susy} searches assume specific simplified models and are thus only valid in the context of the assumptions made in these models.
In any realistic \gls{susy} scenario, assumptions like 100\% branching ratios or small sets of participating, non-decoupled supersymmetric particles are, however, most likely not exactly fulfilled.
Thus, simplified model limits can in general not be trivially interpreted as the true underlying constraints on the respective parameters of a realistic \gls{susy} scenario.
 
Due to the rapidly changing landscape of models for physics beyond the \gls{sm} and the limited scope of parameter limits quoted by the experiments, reinterpretations of searches for supersymmetry are highly desirable and see significant interest from both the experimental and theory communities.
With this in mind, the search for \gls{susy} presented herein was implemented to be fully reinterpretable in the light of new \gls{bsm} models.
This is achieved by using a cyber-infrastructure called \textsc{Recast}~\cite{RECAST_cranmer}, relying on containerised workflows orchestrating parametrised job templates.
Additionally, the full likelihood of the search was made publicly available in a readily available format, allowing it to be incorporated in a number of reinterpretation efforts outside of ATLAS~\cite{SModelS_pyhf:2020grj,Goodsell:2020ddr}. 
 
Large-scale reinterpretations in high-dimensional model spaces are especially interesting, but computationally extremely challenging, and thus require suitable approximations. In this thesis, a method to generically approximate the likelihoods of \gls{susy} searches using binned distributions was introduced and validated using a selection of ATLAS searches for \gls{susy}.
The search previously presented was reinterpreted in the \gls{pmssm}, a 19-dimensional parameter space containing more realistic \gls{susy} scenarios (compared to simplified models). Due to the assumption of 100\% branching fractions not being satisfied in many of these more complete \gls{susy} scenarios, the sensitivity of the \onelepton search was found to be noticeably reduced. A small fraction of models sampled could, however, still be excluded.

The impact of the \onelepton search on electroweakino masses was investigated, revealing some sensitivity to $\tilde{\chi}_2^\pm\neutr$ production with a wino-like \gls{lsp}, in addition to sensitivity towards models phenomenologically close to the simplified model the search was optimised for.
The impact of the \onelepton search on the \gls{dm} relic density was also investigated.
While no conclusive statement could be made for models with a bino-like $\lsp$ because of the limited number of such models sampled in the relevant parameter space, some models with a wino-like $\lsp$ with cosmological abundance satisfying the Planck constraint could still be excluded. 
 
 Although hopes of quickly finding supersymmetric particles with the \gls{lhc} have not materialised, there is still a possibility of finding hints for physics beyond the \gls{sm} in the collision data recorded by the \gls{lhc} experiments.
 Considerable regions of the parameter space of realistic \gls{susy} scenarios are still largely unconstrained and offer ample space for \gls{susy} to hide in.
 In order to provide a comprehensive overview of the constrained parameter space, it is not only important to optimise searches to be sensitive to the complex phenomenology of realistic supersymmetric scenarios, but also to design the searches to be systematically reinterpretable, especially in light of more complete and realistic scenarios.
 After all, searches for \gls{bsm} physics are the tools that shine a light on the otherwise unilluminated landscapes of the parameter spaces of \gls{bsm} theories.
 Allowing these tools to be reusable significantly increases the area of parameter space they can shine a light onto.    
 