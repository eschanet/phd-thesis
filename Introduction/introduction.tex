%!TEX root = ../thesis.tex
%*******************************************************************************
%*********************************** Introduction *****************************
%*******************************************************************************

\chapter*{Introduction}
\markboth{Introduction}{}
\ifpdf
    \graphicspath{{Chapter1/Figs/Raster/}{Chapter1/Figs/PDF/}{Chapter1/Figs/}}
\else
    \graphicspath{{Chapter1/Figs/Vector/}{Chapter1/Figs/}}
\fi

Particle physics studies the fundamental constituents and interactions of matter with the goal of uncovering the laws of nature that ultimately govern the most fundamental building blocks of the universe. Over the course of more than a century, fundamental physics has continuously pushed the intellectual frontier to new realms, reaching ever-smaller length-scales on which the fundamental interactions of the building blocks of matter can be understood. The resulting theoretical framework, the \gls{sm}, provides answers to some of the deepest questions that can be asked about the universe and is the most fundamental---experimentally confirmed---description of nature known to date. 

Particle physics finds itself, however, at an interesting crossroad. On the hand, the \gls{sm} is extraordinarily successful in describing nature at its smallest scales and---with the discovery of the Higgs boson in 2012---has recently been completed. Through various particle physics experiments, the precision and predictive power of the \gls{sm} has been tested to an unprecedented level, finding no significant deviations in experimental data. On the other hand, however, various cosmological and astrophysical observations are putting increasing pressure on the \gls{sm}. Although the existence of \gls{dm} is nowadays well-established, it cannot be suitably described within the theoretical framework of the \gls{sm}. It has become increasingly clear, that the \gls{sm} is an effective theory and thus only an approximation to a more fundamental theory of nature.

A plethora of theories aiming to explain the shortcomings of the \gls{sm} exist. One class of such theories is \gls{susy}, extending the \gls{sm} by associating supersymmetric partners to the \gls{sm} particles. \gls{susy} could, for example, be able to provide a candidate particle for \gls{dm} or serve as a basis for a coherent theory describing all four known fundamental forces. Up until the discovery of the Higgs boson, particle physics was in a state of \textit{symbiosis} where the theory and experimental communities showed each other where to look and what to think about next. Particle physics always had a clear pathway to follow: validate and complete the \gls{sm}. This is, however, no longer the case and experimental particle physics faces an era where a large number of models for physics \gls{bsm} that need to be investigated are constantly being devised by the theory community with no clear indication on where to start looking. 

Although theoretical arguments suggest supersymmetric particles could exist at the energies accessible with the \gls{lhc}, no such particles have been found so far. Up until recently, searches for \gls{susy} have however mostly focused on the production of the supersymmetric partners of quarks and gluons through the strong interaction. With the second run of the \gls{lhc} recently come to an end, an unprecedented amount of collision data has been recorded by the \gls{lhc} experiments and is available for physics analysis. This allows to search for supersymmetric particles produced through the electroweak interaction that have previously not been accessible due to their low theoretical production rates compared to those produced through the strong interaction.

Due to their complexity and lifetimes approaching half a century, experiments like the ATLAS detector at the \gls{lhc} severely challenge the scientific method as they are in general not easily repeatable. This precarious situation, coupled with the ever-changing landscape of promising \gls{bsm} models developed by the theory community, requires efforts to not only preserve searches for \gls{bsm} physics, but make them fully re-usable in the context of new, promising \gls{bsm} models. 

This thesis presents a search for the supersymmetric partners of the \gls{sm} Higgs and gauge bosons, collectively referred to as \textit{electroweakinos}. The search uses \onethirtynineifb of proton--proton collision data recorded at a centre-of-mass energy of $\SI{13}{\TeV}$ using the ATLAS detector. The search is embedded in a larger effort within the ATLAS collaboration searching for \gls{susy} using a variety of theoretical models. The thesis is divided into four main parts. In \cref{part:fundamentals}, the fundamental concepts necessary for the remainder of the thesis are presented. This includes a theoretical introduction to the \gls{sm} and \gls{susy}, followed by a description of the experimental setup, concluding with a discussion of the statistical concepts used. \Cref{part:simplified_model_analysis} introduces the aforementioned search for electroweakinos and discusses its results using \onethirtynineifb of data recorded by ATLAS. In \cref{part:reinterpretation}, preservation and re-usability efforts are presented, aiming to significantly increase the scientific impact of the search by making it readily available to re-interpretation efforts inside as well as outside of the ATLAS collaboration. Additionally, a method for approximating the statistical models of \gls{susy} searches is introduced and discussed, followed by a re-interpretation in a high-dimensional parameter space containing complex \gls{susy} scenarios. Finally, the thesis concludes with a brief summary in \cref{part:summary}.

\improvement{Natural units and Minkowski metric}