%!TEX root = ../thesis.tex
%*******************************************************************************
%*********************************** Introduction *****************************
%*******************************************************************************

\chapter*{Introduction}
\markboth{Introduction}{}
\ifpdf
    \graphicspath{{Chapter1/Figs/Raster/}{Chapter1/Figs/PDF/}{Chapter1/Figs/}}
\else
    \graphicspath{{Chapter1/Figs/Vector/}{Chapter1/Figs/}}
\fi

Particle physics studies the fundamental constituents and interactions of matter with the ultimate goal of uncovering the laws of nature that govern the most fundamental building blocks of the universe.
Over the course of more than a century, fundamental physics has continuously pushed the frontiers of knowledge, reaching ever-smaller length-scales on which the fundamental interactions of the building blocks of matter can be understood.
The resulting theoretical framework, the \gls{sm}, provides answers to some of the deepest questions that can be asked about the universe and is the most fundamental, experimentally validated description of nature known to date. 

Particle physics finds itself, however, at an interesting crossroad. On the one hand, the \gls{sm} is very successful in describing nature at its smallest scales and---with the discovery of the Higgs boson in 2012~\cite{HIGG-2012-27,CMS-HIG-12-028}---has recently been experimentally completed.
Through various particle physics experiments, the precision and predictive power of the \gls{sm} have been tested to an unprecedented level, finding no significant deviations in experimental data so far.
On the other hand, however, a number of cosmological observations as well as flavour and precision electroweak measurements are putting increasing pressure on the \gls{sm}.
For example, although the existence of \gls{dm} is nowadays well-established, it cannot be suitably described within the theoretical framework of the \gls{sm}.
Over the course of the last decades, it has become increasingly clear that the \gls{sm} is an effective theory, and thus only a low-energy approximation to a more fundamental theory of nature.

A plethora of theories aiming to explain the shortcomings of the \gls{sm} exist. One class of such theories is \gls{susy}, extending the \gls{sm} by associating supersymmetric partners to the \gls{sm} particles.
\gls{susy} could, for example, be able to provide a candidate particle for \gls{dm} or explain some of the tensions observed in electroweak precision measurements.
Up until the discovery of the Higgs boson, the theory and experimental communities in particle physics were in a state of \textit{symbiosis} with a clear pathway to follow: validating and completing the \gls{sm}. This is, however, no longer the case and experimental particle physics faces an era where a large number of models for \gls{bsm} physics can be thought of, but no clear indication of where to start looking is available. 

Although theoretical arguments suggest that supersymmetric particles could exist at the energies accessible with the \gls{lhc}, no such particles have been found so far.
Up until recently, searches for \gls{susy} have, however, mostly focused on the production of the supersymmetric partners of quarks and gluons through the strong interaction.
With the second run of the \gls{lhc} recently come to an end, an unprecedented amount of proton--proton collision data has been recorded by the \gls{lhc} experiments and is available for physics analysis.
This allows to search for supersymmetric particles produced through the electroweak interaction that have previously not been accessible due to their low theoretical production rates, compared to those produced through the strong interaction.

Due to their complexity and lifetimes approaching half a century, experiments like the ATLAS detector at the \gls{lhc} are in general not easily repeatable and thus severely challenge the scientific method.
This precarious situation, coupled with the wide landscape of \gls{bsm} models available to search for, requires efforts to not only preserve searches for \gls{bsm} physics, but make them fully reusable in the context of new, additional \gls{bsm} models. 

This thesis presents a search for the supersymmetric partners of the \gls{sm} Higgs and gauge bosons, collectively referred to as \textit{electroweakinos}. The search uses \onethirtynineifb of proton--proton collision data recorded at a centre-of-mass energy of $\SI{13}{\TeV}$ with the ATLAS detector.
It is embedded in a larger effort within the ATLAS collaboration, searching for \gls{susy} in the context of a variety of theoretical models.
The present work is divided into four main parts.
In \cref{part:fundamentals}, the fundamental concepts necessary for the remainder of the thesis are presented.
This includes a theoretical introduction to the \gls{sm} and \gls{susy}, followed by a description of the experimental setup, concluding with a discussion of the statistical concepts used.
\Cref{part:simplified_model_analysis} introduces the aforementioned search for electroweakinos and discusses its results using \onethirtynineifb of proton--proton collision data recorded by ATLAS.
In \cref{part:reinterpretation}, preservation and reusability efforts are presented, aiming to make the search readily available to reinterpretation efforts inside as well as outside of the ATLAS collaboration.
Furthermore, a method for approximating the statistical models of \gls{susy} searches is introduced and validated in \cref{ch:simplify}. These efforts culminate in a reinterpretation of the search in a subspace of a 19-dimensional set of more complete supersymmetric scenarios, the results of which are discussed in \cref{ch:pmssm}.
Finally, the thesis concludes with a brief summary in \cref{part:summary}.

\improvement{Natural units and Minkowski metric}