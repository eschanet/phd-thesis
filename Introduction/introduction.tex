%!TEX root = ../thesis.tex
%*******************************************************************************
%*********************************** Introduction *****************************
%*******************************************************************************

\chapter*{Introduction}
\markboth{Introduction}{}
\ifpdf
    \graphicspath{{Chapter1/Figs/Raster/}{Chapter1/Figs/PDF/}{Chapter1/Figs/}}
\else
    \graphicspath{{Chapter1/Figs/Vector/}{Chapter1/Figs/}}
\fi

Elementary particle physics studies the fundamental constituents and interactions of matter with the goal of uncovering the laws of nature that ultimately govern the most fundamental building blocks of the universe. Over the course of more than a century, fundamental physics has continuously pushed the intellectual frontier to new realms, reaching ever-smaller length-scales on which the fundamental interactions of the building blocks of matter can be understood. The resulting theoretical framework, the \gls{sm}, provides answers to some of the deepest questions that can be asked about the universe and is the most fundamental description of nature to date. Through various experiments, the precision and predictive power of the \gls{sm} has been tested to an unprecedented level, finding no significant deviations in experimental data.

In order to reach the length-scales needed to resolve and study the innermost structures of matter, particle physics relies on high energetic particle collisions whose outcomes are recorded in dedicated particle detectors. With the goal of resolving increasingly small length-scales, the energies accessible in the particle collisions have continuously been pushed to new frontiers, necessitating increasingly complex machines. Today, particle physics experiments are complex machines requiring large international collaborations of scientists and engineers to operate them and analyse the data. Due to their complexity and lifetimes approaching half a century, such experiments severely challenge the scientific method as they are in general not readily repeatable. Special care is thus required in order to preserve the efforts analysing the data recorded. Arguably the most complex particle accelerator nowadays is the \gls{lhc}, operated by CERN in Geneva, Switzerland. From 2015--2018, \gls{lhc} has generated particle collisions at an unprecedented centre-of-mass energy of $\SI{13}{\TeV}$, recorded by large-scale detectors like the ATLAS experiment.

Although extraordinarily successful in describing nature at its smallest scales, it has become increasingly obvious that the \gls{sm} cannot be a complete theory of nature. The \gls{sm} only describes three of the four known fundamental forces of the universe, namely the electromagnetic, weak and strong interactions. Gravity, the last of the four fundamental forces, cannot be described at all in the theoretical framework of the \gls{sm}. Furthermore, the \gls{sm} is not able to suitably describe \gls{dm}, even though its existence is nowadays considered to be well-established due to various astronomical and astrophysical observations. It seems then that the \gls{sm} is only a low-energy approximation of a more fundamental theory of nature.

A plethora of theories aiming to explain the shortcomings of the \gls{sm} exist. One class of such theories is \gls{susy}, extending the \gls{sm} by associating supersymmetric partners to the \gls{sm} particles. \gls{susy} could, for example, be able to provide a candidate particle for \gls{dm} or serve as a basis for a coherent theory describing all four fundamental forces, \ie bringing gravity into the picture. Although theoretical arguments suggest supersymmetric particles could exist at the energies accessible with the \gls{lhc} no supersymmetric particles have been found so far. Up until recently, searches for \gls{susy} have however mostly focused on production of the supersymmetric partners of quarks and gluons through the strong interaction. With the dataset recorded by ATLAS from 2015--2018, an unprecedented amount of collision data is available for physics analysis, allowing to search for supersymmetric particles produced through the electroweak interaction that have previously not been accessible due to their comparably low theoretical production rate.

This thesis presents a search for the supersymmetric partners of the \gls{sm} Higgs boson and gauge bosons, collectively referred to as \textit{electroweakinos}, with the ATLAS experiment. The thesis is divided into four main parts. In \cref{part:fundamentals}, the fundamental concepts necessary for the remainder of the thesis are presented. This includes a theoretical introduction to the \gls{sm} and \gls{susy}, followed by a description of the experimental setup and finished up by a discussion of the statistical concepts used. \Cref{part:simplified_model_analysis} introduces the aforementioned search for electroweakinos and discusses its results using \onethirtynineifb of data recorded by ATLAS. In \cref{part:reinterpretation}, efforts towards a fully re-usable and re-interpretable search are presented, and a method for approximating the statistical models of searches for \gls{susy} is discussed. This is followed by a re-interpretation in a high-dimensional parameter space containing realistic \gls{susy} scenarios, illustrating the computational feasibility using the preceding approximations. Finally, the thesis concludes with a brief summary in \cref{part:summary}.

\improvement{Natural units and Minkowski metric}