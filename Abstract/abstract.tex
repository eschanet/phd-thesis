% ************************** Thesis Abstract *****************************
% Use `abstract' as an option in the document class to print only the titlepage and the abstract.

\sisetup{output-decimal-marker = {,}}
\begin{otherlanguage}{ngerman} 
\begin{zusammenfassung}

Obwohl das Standardmodell der Teilchenphysik eine außerordentlich erfolgreiche Theorie darstellt, deuten einige Beobachtungen auf die Existenz neuer Physik jenseits dessen was im Rahmen des Standardmodells erklärt werden kann hin.
Supersymmetrie ist der Oberbegriff für eine Klasse von Theorien, die einige der offenen Fragen des Standardmodells erklären könnten.
Sie sagt die Existenz von supersymmetrischen Partnern für jedes Teilchen des Standardmodells voraus und könnte, unter anderem, einen Teilchenkandidaten für dunkle Materie liefern. 

Diese Arbeit stellt eine Suche nach supersymmetrischen Teilchen, die über die elektroschwache Wechselwirkung paarproduziert werden, vor.
Endzustände mit einem Lepton, fehlender Transversalenergie und einem Higgs Boson, welches in zwei \textit{b}-Quarks zerfällt, werden untersucht.
Insgesamt werden \onethirtynineifb an Daten aus Proton-Proton Kollisionen berücksichtigt, welche mit dem ATLAS Detektor bei einer Schwerpunktsenergie von $\sqrt{s}=\SI{13}{\TeV}$ im Run~2 des Large Hadron Colliders aufgezeichnet wurden.
Ein, auf einer Likelihood-Methode basierender, simultaner Fit in allen Suchregionen wird verwendet, um hohe Sensitivität zu möglichst vielen kinematischen Bereichen im untersuchten Parameterraum zu gewährleisten.

Keine signifikante Abweichung von den Standardmodellvorhersagen wird in den Daten beobachtet, weshalb die Ergebnisse in einem vereinfachten Modell für Paarproduktion von Elektroweakinos interpretiert werden.
Leichteste Charginos und zweitleichteste Neutralinos mit Massen bis zu $\SI{740}{\GeV}$ ($\SI{600}{\GeV}$) können für leichteste Neutralino Massen von $\lesssim \SI{100}{\GeV}$ ($\approx\SI{250}{\GeV}$) ausgeschlossen werden.
%Die Ausschlussgrenzen vorheriger ATLAS Suchen nach Supersymmetrie werden signifikant verbessert.

Da heutige Teilchenphysik-Experimente aufgrund ihrer Komplexität und Größenordnung nicht trivial reproduzierbar sind, gleichzeitig aber eine Vielzahl an Modellen für Physik jenseits des Standardmodells existiert, wird ein besonderes Augenmerk auf die technische Durchführbarkeit gelegt, die Suche in neuen Modellen zu interpretieren.
Die volle Likelihood-Funktion der Suche wird veröffentlicht und eine vollständig reproduzierbare Umsetzung der Suche anhand Container-Technologie und parametrisierter Job-Vorlagen wird diskutiert.
Mit Hinblick auf rechenintensive Neuinterpretationen in hoch-dimensionalen Parameterräumen wird eine Methode eingeführt um die Likelihood-Funktionen von ATLAS Suchen nach Supersymmetrie generisch zu nähern.
Mit Hilfe dieser Methode wird schlussendlich eine Neuinterpretation der Suche in einem Unterraum einer 19-dimensionalen Menge von vollständigeren supersymmetrischen Modellen durchgeführt und deren Ergebnisse diskutiert. 

\end{zusammenfassung}
\end{otherlanguage}  
\sisetup{output-decimal-marker = {.}}

\begin{abstract}

Despite the success of the Standard Model of Particle Physics, a number of hints suggest the existence of new physics beyond the scope of phenomena that can be explained in the theoretical framework of the Standard Model.
One class of theories that could be able to explain some of the open questions of the Standard Model is Supersymmetry. It introduces supersymmetric partners to each of the Standard Model particles and could, for example, provide a candidate for Dark Matter.

This thesis presents a search for electroweak production of supersymmetric particles in events with one lepton, missing transverse momentum and a Higgs boson decaying into two \textit{b}-quarks.
The search analyses \onethirtynineifb of proton--proton collision data at a centre-of-mass energy of $\sqrt{s}=\SI{13}{\TeV}$, recorded by the ATLAS detector.
A two-dimensional shape-fit is introduced in order to achieve sensitivity to a large variety of kinematic regimes.
No significant deviation from the Standard Model predictions are seen in data in any of the search regions.
The results are subsequently interpreted in a simplified model for electroweakino pair-production.
Lightest chargino and next-to-lightest neutralino masses of $\SI{740}{\GeV}$ ($\SI{600}{\GeV}$) can be excluded for lightest neutralino masses of $\lesssim \SI{100}{\GeV}$ ($\approx\SI{250}{\GeV}$).
%, significantly improving the limits set by previous ATLAS searches for Supersymmetry.

Given that today's particle physics experiments are not easily reproducible and a large number of phenomenologically viable models for physics beyond the Standard Model exist, special focus is put on the reusability and reinterpretability of the search.
The full likelihood function of the search is published in a readily available format, and a fully reusable implementation of the search using containerised workflows with parameterised job templates is provided. 
In light of conceptually interesting  but computationally challenging reinterpretations in high-dimensional model spaces, a method for generically approximating the likelihood functions of searches for Supersymmetry is introduced and validated. Using this approach, a reinterpretation of the search in a subspace of a 19-dimensional set of more complete supersymmetric models is performed and its results are discussed. 

\end{abstract}


